\documentclass[letterpaper, 11pt]{article}
\usepackage[utf8]{inputenc}

\usepackage{setspace, longtable, graphicx, hyphenat, hyperref, fancyhdr, ifthen, enumitem, amsmath, setspace}

% Set page margins
\usepackage[left=1in, right=1in, bottom=0.7in, top=0.7in]{geometry}

% Set line spacing
\renewcommand{\baselinestretch}{1.15}

% --- Page formatting ---

% Set link colors
\usepackage[dvipsnames]{xcolor}
\newcommand{\link}[2]{{\color{blue}\href{#1}{#2}}}

% Set font to Libertine, including math support
\usepackage{libertine}
\usepackage[libertine]{newtxmath}

% Remove page numbering
\pagenumbering{gobble}

% --- Document starts here ---

\begin{document}

% Name and date of last update to this document
\noindent{\Huge{Floyd Everest}
\hfill{\it\footnotesize Updated \today}}

% --- Contact information and other items ---

\vspace{0.5cm} 
\begin{center}
\begin{tabular}{ll}
% Line 1: Email, GitHub, office location
\textbf{Email}: floyd.everest@monash.edu &
\hspace{0.4in} \textbf{GitHub}: \link{https://github.com/fleverest}{fleverest} \\
% Line 2: Phone number, LinkedIn, citizenship
\textbf{Phone}: (+61) red-act-ed!   & 
\hspace{0.4in} \textbf{Citizenship}: Australia
\end{tabular} \\
\hfill\textbf{Interests}: Decision Theory, Sequential Analysis, Open-Source Software,\ \ \ \ \ \ \ \ \ \ \ \ \ \ \ \ \ \ \ \ \ \ \ \ \ 
\end{center}

% --- Start the two-column table storing the main content ---

% Set spacing between columns
\setlength{\tabcolsep}{8pt}

% Set the width of each column
\begin{longtable}{p{0.8in}p{5.4in}}

% --- Section: Education ---

\color{OliveGreen}{Education}
& \textbf{Monash University} (PhD) \hfill Clayton, Victoria \\
& Doctor of Philosophy -- Econometrics and Business Statistics \hfill August 2023 -- TBD \\
& Supervisor: A. Prof. Damjan Vukcevic \\

& \textbf{University of Melbourne} \hfill Parkville, Victoria \\ 
& Master of Data Science  \hfill March 2021 -- August 2022 \\
& Thesis: Bayesian Auditing of IRV Elections with Dirichlet-tree priors \\

& \textbf{University of Melbourne} \hfill Parkville, Victoria \\ 
& Graduate Diploma in Data Science \hfill March 2020 -- December 2020 \\

& \textbf{University of Melbourne} \hfill Parkville, Victoria \\ 
& Bachelor of Science (Pure Mathematics) \hfill March 2015 -- December 2017 \\
& \\

\color{OliveGreen}{Publications}
& \link{https://doi.org/10.1007/978-3-031-25460-4_30}{\textbf{Ballot-Polling Audits of IRV Elections with a Dirichlet-Tree Model}} \\
& F. Everest, M. Blom, P.B. Stark, P.J. Stuckey, V. Teague, D. Vukcevic. \\
& In proceedings of EIS 2022. \\
& \link{https://doi.org/10.48550/arXiv.2206.14605}{\textbf{Auditing Ranked Voting Elections with Dirichlet-Tree Models: First Steps}} \\
& F. Everest, M. Blom, P.B. Stark, P.J. Stuckey, V. Teague, D. Vukcevic. \\
& \\

% --- Section: Work experience ---
\nohyphens{\color{OliveGreen}{Past Work}} 
& \textbf{University of Melbourne (Mathematics and Statistics)} \\
& Supervisors: Damjan Vukcevic and Michelle Blom \hfill September 2020 -- March 2023 \\
& I implemented a Bayesian model for modelling Election outcomes in R (available as an R package on CRAN: \link{https://cran.r-project.org/package=elections.dtree}{elections.dtree}) and investigated the model's performance using real Australian Election data. This involved running distributed Monte-Carlo simulations on the Spartan HPC facility. I also worked full-time as a Research Assistant for 7 weeks from January to March (2022). This is when I published my second R package \link{https://cran.r-project.org/package=prefio}{\textbf{prefio}}, which simplifies working with preferential data in R.\\
 
& \textbf{Melbourne Business School} \\
& Supervisor: Michael Smith \hfill March 2021 -- November 2022 \\
& I worked as a casual research assistant during my Master of Data Science. I implemented a novel MCMC regression model for endogenous data in R \& C++.\\

& \textbf{Backend Developer \& Systems Administrator: DuxTel pty. ltd. } \\
& \hfill January 2018 -- December 2022 \\
& I worked as a Backend Developer and Sysadmin for 4 years during my studies. Full-time between semesters and part-time during. My main contribution was to migrate away from legacy platforms (self-hosted ecommerce and ticketing systems).\\
& \\

\pagebreak

\color{OliveGreen}{Presentations}
& \textbf{Auditing Ranked Voting Elections with Dirichlet-Tree Models}\\
& \textit{ECSSMini 2022} \\
& I presented my work in a 15 minute session for the 2022 ECSS (Early Career \& Student Statisticians) network conference. This conference was hybrid-virtual and I presented via zoom to a live audience in Perth.\\
& \textbf{Auditing ranked voting elections with Dirichlet-tree models: first steps}\\
& \textit{E-Vote-ID 2022} \\
& I presented a 10 minute talk along with my supervisor at the 2022 E-Vote-ID conference in Bregenz, Austria. I was privileged to present our work on Election Auditing in person, for over 100 physical attendants.\\
& \textbf{Ballot-Polling Audits of Instant-Runoff Voting Elections with a Dirichlet-Tree Model}\\
& \textit{ESORICS 2022} \\
& I presented along with my supervisor for the 2022 ESORICS workshop on Election Infrastructure and Security. We presented for 15 minutes virtually via zoom.\\
& \\

\color{OliveGreen}{My Projects}
& \textbf{\link{https://cran.r-project.org/package=prefio}{\textbf{prefio}}} - R\\
& A library for processing and working with Ordinal Preference datasets which contain rankings on a fixed set of items. I was motivated to create this library while processing election data from the NSW Electoral Commission website for use in our Election Auditing research project. \\
& \textbf{\link{https://cran.r-project.org/package=elections.dtree}{elections.dtree}} - R, C++\\
& A package for conducting Ballot Polling Bayesian Audits with Dirichlet-tree priors. This is an implementation of the model I developed for my Master of Data Science research project. It models the distribution of ranked ballots cast in an election, which lies in a very high-dimensional space. For example in the Australian House of Representatives elections we are required to number $k$ candidates from 1 to $k$ with no repeats: there are $k!$ valid ways of doing this. To model $k!$ different proportions, we use a Dirichlet-tree to break up the space into nested subspaces, which improves the efficiency of statistical inference and allows for many computational shortcuts.\\
& \\

% --- Section: Various skills (programming, software, languages, etc.) ---
{\color{OliveGreen}{Skills}} 
& \textbf{Programming \& Data Analysis}\\
& \textit{Extensive experience with: }R, Python, C/C++, Excel, SQL, Git\\
& \textit{Familiar with: }Haskell, Java, JavaScript, Bash, Prolog, Ruby\\
& \textbf{Automation, Infrastructure and Networking}\\
& \textit{Extensive experience with: }GNU/Linux, K8s, Containers, Linux Networking (L2/3), NGINX, Slurm, GitLab CI/CD\\
& \textit{Familiar with: }Ansible, Terraform, CRON, S3, OpenFaaS, AWS Lambda\\
& \\

% --- End of CV! ---

\end{longtable}
\end{document}